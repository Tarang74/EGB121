%!TEX program = xelatex
\documentclass{article}
\usepackage{LaTeX-Submodule/template}

% Additional packages & macros

% Header and footer
\newcommand{\unitName}{Engineering Mechanics}
\newcommand{\unitTime}{Semester 1, 2022}
\newcommand{\unitCoordinator}{Dr Tuquabo Tesfamichael}
\newcommand{\documentAuthors}{\textsc{Tarang Janawalkar}}

\fancyhead[L]{\unitName}
\fancyhead[R]{\leftmark}
\fancyfoot[C]{\thepage}

% Copyright
\usepackage[
    type={CC},
    modifier={by-nc-sa},
    version={4.0},
    imagewidth={5em},
    hyphenation={raggedright}
]{doclicense}

\date{}

\begin{document}
%
\begin{titlepage}
    \vspace*{\fill}
    \begin{center}
        \LARGE{\textbf{\unitName}} \\[0.1in]
        \normalsize{\unitTime} \\[0.2in]
        \normalsize\textit{\unitCoordinator} \\[0.2in]
        \documentAuthors
    \end{center}
    \vspace*{\fill}
    \doclicenseThis
    \thispagestyle{empty}
\end{titlepage}
\newpage
%
\tableofcontents
\newpage
%
\section{Stress and Strain}
\subsection{External Forces}
Rigid bodies are subjected to external force and couple moment systems that result from the effects of gravitational,
electrical, magnetic, or contact forces. Contact forces can be surface, linear, or concentrated forces.
\subsubsection{Types of Forces}
\begin{itemize}
    \item Compressive (pushing)
    \item Tensile (pulling)
    \item Shear (sliding)
    \item Torsional (twisting)
    \item Biaxial tension
    \item Hydrostatic compression
    \item Bending (induces tension, compression and shear)
\end{itemize}
\subsection{Internal Loadings}
External forces cause internal loadings that occur in equal and opposite collinear pairs as stresses and strains.
Internal loading is associated with \textbf{stress} while \textbf{strain} is a measure of a body's deformation.

These loadings have no external effects on the body, and are not included on a \textbf{Free Body Diagram} (FBD) if
the entire body is considered.

To determine the forces in each member, we can use the method of sections to represent the internal loading as external forces.
\subsection{Internal Resultant Loadings}
Although the exact distribution of the internal loading may be \textit{unknown}, we can determine the
resultant force \(\symbf{F}_R\) and resultant moment \(\left( \symbf{M}_R \right)_O\) about a point \(O\) by applying the
equations of equilibrium
\begin{align*}
    \sum \symbf{F}   & = \symbf{0}  \\
    \sum \symbf{M}_O & = \symbf{0}.
\end{align*}
\begin{figure}[H]
    \centering
    \includegraphics[height = 8cm, keepaspectratio = true]{figures/resultant_loadings.pdf}
    \caption{Resultant loadings acting on a body.}
    % \label{}
\end{figure}
\subsubsection{In 3D}
In 3D, we can represent resultant loadings using four vectors acting over the sectioned area.
\begin{description}
    \item[Normal force] \(\symbf{N}\) force acting perpendicular to the area
    \item[Shear force] \(\symbf{V}\) force acting on an axis tangent to the area
    \item[Torsional moment] \(\symbf{T}\) rotation about the perpendicular axis
    \item[Bending moment] \(\symbf{M}\) rotation about an axis tangent to the area
\end{description}
\subsubsection{In 2D}
In 2D, the body is subjected to a coplanar system of forces, where \(\symbf{T} = \symbf{0}\).
\subsubsection{In 1D}
In 1D, the body is only subjected to axial forces, where \(\symbf{V} = \symbf{T} = \symbf{M} = \symbf{0}\).
\subsection{Stress}
The force and moment acting at a specific point on a sectioned area of a body represent the resultant effects of
the distribution of internal loading that acts over the sectioned area.
\begin{definition}[Stress]
    Consider the quotient of the force \(\Delta \symbf{F}\) over an area \(\Delta A\), then as the \(\Delta A \to 0\),
    so does \(\Delta \symbf{F}\), while the quotient approaches a finite limit. This quotient is called the stress at
    that point.
    \begin{equation*}
        \symbfit{\sigma} = \lim_{\Delta A \to 0} \frac{\Delta \symbf{F}}{\Delta A}
    \end{equation*}
    Here the normal and shear stresses can be expressed using \(\sigma_z\) and \(\tau_{zx}\) and \(\tau_{zy}\).
    \begin{align*}
        \sigma_z  & = \lim_{\Delta A \to 0} \frac{\Delta F_z}{\Delta A} \\
        \tau_{zx} & = \lim_{\Delta A \to 0} \frac{\Delta F_x}{\Delta A} \\
        \tau_{zy} & = \lim_{\Delta A \to 0} \frac{\Delta F_y}{\Delta A}
    \end{align*}

    Stress describes the intensity of the internal force acting on a specific region passing through a point.

    The unit for stress is Pascal where \qty{1}{Pa} or \qty{1}{N.m^{-2}} and \qty{1}{MPa} or \qty{1}{N.mm^{-2}}.
\end{definition}
\subsection{Average Normal Stress}
To determine the average stress distribution acting over a cross-sectional area of an axially loaded bar,
we assume that the material is both \textit{homogeneous} and \textit{isotropic}.
This means the load \(P\) applied through the centroid of the cross-sectional area
will cause the bar to deform uniformly throughout the central region of its length.

By passing a section through a bar, equilibrium requires the resultant normal force \(N\) at the section to be
equal to the external force \(P\). And because the material undergoes a uniform deformation, it is necessary
that the cross section be subjected to a constant normal stress distribution.

As a result, each small area \(\Delta A\) on the cross section is subjected
to a force \(\Delta N = \sigma \Delta A\), where the sum of these forces over the entire cross-sectional area
is \(P\). By letting \(\Delta A \to \odif{A} \) and therefore also \( \Delta N \to \odif{N} \), then as \(\sigma\)
is a constant, we have
\begin{align*}
    \int \odif{N} & = \int_A \sigma \odif{A} \\
    N             & = \sigma A
\end{align*}
Therefore
\begin{equation*}
    \sigma_{\mathrm{avg}} = \frac{N}{A}
\end{equation*}
where in this case \(N = P\).
\begin{theorem}[Equilibrium]
    For an uniaxially loaded body, the equation of force equilibrium gives
    \begin{align*}
        \sigma \left( \Delta A \right) - \sigma' \left( \Delta A \right) & = 0       \\
        \sigma                                                           & = \sigma'
    \end{align*}
    hence the normal stress components must be equal in magnitude but opposite in direction.

    Under this condition, the material is subjected to \textbf{uniaxial stress} and this analysis
    applies to members subjected to tension or compression.
\end{theorem}
\subsection{Strain}
\begin{definition}[Deformation]
    Whenever a force is applied to a body, it will tend to change the body's shape and size.
    These changes are referred to as deformation.
\end{definition}
\begin{definition}[Strain]
    To describe the deformation of a body through changes in lengths of line segments on the surface,
    we will develop the concept of strain.
    If an axial load \(P\) is applied to a bar, it will change the bar's length \(L_0\) to \(L\).
    Then the \textbf{average normal strain} of the bar is defined
    \begin{equation*}
        \epsilon_{\mathrm{avg}} = \frac{L - L_0}{L_0}
    \end{equation*}
    where the numerator is often written as \(\delta = L - L_0\) and is known as elongation or extension.

    The \textbf{normal strain} \(\epsilon\) at a point in a body with an arbitrary shape is defined similarly.
    Consider a small line segment \(\Delta s\) which becomes \(\Delta s'\) after deformation. Then the limit of the normal strain
    is
    \begin{equation*}
        \epsilon = \lim_{\Delta s \to 0} \frac{\Delta s' - \Delta s}{\Delta s}
    \end{equation*}
    In both cases normal strain is positive when the initial length elongates, and negative when the length contracts.

    Strain is a dimensionless quantity sometimes expressed \unit{mm/mm} or \unit{m/m}, or as a percentage.
\end{definition}
\section{Tension and Compression Tests}
To determine the strength of a material, we must perform a tension or commpression test.
This test measures the stress and strain from a load \(P\), and the results can be used to
produce a \textbf{stress-strain diagram}.
There are two ways in which the stress-strain diagram is normally described.
\subsection{Stress-Strain Diagram}
\begin{figure}[H]
    \centering
    \includegraphics[height = 8cm, keepaspectratio = true]{figures/stress_strain_diagram.pdf}
    \caption{Stress-strain diagram for a typical metal.}
    % \label{}
\end{figure}
\subsubsection{Conventional Stress-Strain Diagram}
The engineering stress assumes that the area \(A\) is constant throughout
the gauge length
\begin{equation*}
    \sigma = \frac{P}{A_0}
\end{equation*}
where \(A_0\) is the \textit{original} cross-sectional area of the specimen.

Likewise, the engineering strain uses the specimen's original length \(L_0\)
\begin{equation*}
    \epsilon = \frac{\delta}{L_0}
\end{equation*}
\subsubsection{True Stress-Strain Diagram}
The true stress and true strain use the instantaneous area \(A\) and length \(L\)
at each measurement.
\subsection{Elastic Behaviour}
The initial region of the curve is referred to as the \textbf{elastic region}
where the deformation is \textit{elastic}, so that unloading causes 
the specimen to return to its original shape.
\subsubsection{Proportional Limit}
For the majority of the elastic deformation, the curve is \textit{linear}
up to the point where the stress reaches the \textbf{proportional limit} 
at \(\left( \sigma_{pl},\, \epsilon_{pl} \right)\).
\subsubsection{Modulus of Elasticity}
The linear relationship up to this point is characterised by Hooke's law, and is expressed as
\begin{equation*}
    \sigma = E \epsilon
\end{equation*}
where \(E\) is the constant of proportionality, called the \textbf{modulus of elasticity}
or \textbf{Young's modulus}.
\subsection{Elastic Limit}
When the stress slightly exceeds the proportionality limit, 
the curve bends until the stress reaches an \textbf{elastic limit}.
\subsection{Plastic Behaviour}
An increase in stress above the elastic limit will result in a breakdown of the material 
and cause it to deform plastically.
\subsubsection{Yielding}
This behaviour is called \textbf{yielding}
and the stress that causes yielding occurs at the \textbf{yield point} 
\(\left( \sigma_{Y},\; \epsilon_{Y} \right)\). 
Although not shown in the diagram, the yield point is distinguished as two points.

The \textbf{upper yield point} occurs first, followed by a sudden decrease in load-carrying capacity
to a \textbf{lower yield point}. Once the yield point is reached, \textit{the specimen will continue to 
elongate \textbf{without} any increase in load}. When the material behaves in this manner,
it is often referred to as being \textbf{perfectly plastic}.
\subsubsection{Yield Strength}
Commonly the proportionality limit, the elastic limit, and yield point are indistinguishable,
due to this, the \textbf{yield strength} is defined at \(\left( \sigma_{YS},\; \epsilon_{YS} \right)\).

To determine this point, a 0.2\% strain is chosen, and a line with gradient \(E\) is drawn
from the \(\epsilon\) axis.
The point where this line intersects the curve defines \(\left( \sigma_{YS},\; \epsilon_{YS} \right)\).
\subsubsection{Strain Hardening}
Yielding ends when any loading causes the stress
to increase, this rise in the curve is referred to as \textbf{strain hardening}.

When a plastically deformed ductile material is unloaded, 
the elastic strain is recovered as the material returns to its equilibrium state. 

However the plastic strain is maintained, resulting in a \textbf{permanent set}. 
\begin{figure}[H]
    \centering
    \includegraphics[height = 8cm, keepaspectratio = true]{figures/strain_hardening.pdf}
    \includegraphics[height = 8cm, keepaspectratio = true]{figures/strain_hardening_recovery.pdf}
    \caption{Elastic strain recovery under strain hardening.}
    % \label{}
\end{figure}
\subsubsection{Ultimate Tensile Stress}
The maximum stress reached on the diagram is referred to as the \textbf{ultimate tensile stress} \(\left( \sigma_{UTS},\; \epsilon_{UTS} \right)\).
\subsubsection{Necking}
While the specimen elongates up to \(\epsilon_{UTS}\), its cross-sectional area will decrease \textit{uniformally} over its gauge length.
However after reaching \(\epsilon_{UTS}\), the cross-sectional area will decrease \textit{locally}, causing an increase in stress. 
As a result, a ``neck'' forms at this region, and the specimen experiences \textbf{necking}.
\subsubsection{Fracture Stress}
Finally, the specimen breaks where the curve ends at the \textbf{fracture point} at \(\left( \sigma_f,\; \epsilon_f \right)\).
\subsection{Ductility}
\begin{definition}[Ductility]
    Ductility is a measure of the amount of plastic deformation a material can sustain under tensile stress before failure.

    Ductility can be measured using the \textbf{percent elongation} (in length) or \textbf{percent reduction} (in area)
    of a material.
    \begin{align*}
        \textrm{Percent Elongation} = \frac{L_f - L_0}{L_0} \qty{100}{\%} \\
        \textrm{Percent Reduction} = \frac{A_0 - A_f}{A_0} \qty{100}{\%}
    \end{align*}
    As the elastic region is very brief in most materials, ductility is often measured using the 
    original length and area, rather than the length and area when the material undergoes
    plastic deformation.
\end{definition}
\subsection{Brittleness}
\begin{definition}[Brittleness]
    Brittleness describes the property of a material that fractures with little to no yielding.
\end{definition}
\subsection{Poisson's Ratio}
When a deformable body is subjected to a force, it can elongate longitudinally and also contract laterally.
The strain in the longitudinal (or axial) direction is given by
\begin{equation*}
    \epsilon_{\mathrm{long}} = \frac{\delta}{L}
\end{equation*}
and the strain in the lateral (or radial) direction is given by
\begin{equation*}
    \epsilon_{\mathrm{lat}} = \frac{\delta'}{r}
\end{equation*}
where \(\delta'\) is the change in the radius \(r\).

Consider the ratio of these two quantities \(\nu\)
\begin{equation*}
    \nu = -\frac{\epsilon_{\mathrm{lat}}}{\epsilon_{\mathrm{long}}}.
\end{equation*}
Within the elastic region, \(\nu\) will be constant, and it is referred
to as \textbf{Poisson's ratio}.

Note the negative value is introduced as the longitudinal and lateral strains have opposite signs.
\subsection{Strain Energy}
As a material is deformed under external load, the load will do external work.
This work is stored in the material as internal energy or \textbf{strain energy}. 
\begin{figure}[H]
    \centering
    \includegraphics[height = 6cm, keepaspectratio = true]{figures/strain_energy.pdf}
    \caption{Internal energy in small element.}
    % \label{}
\end{figure}
If we consider a small volume element of the material, then the force is equal to the 
average force magnitude \(\Delta F / 2\) and the displacement is given by 
\(d\). Therefore the strain energy \(\Delta U\) is given by
\begin{align*}
    \Delta U & = \frac{1}{2} \Delta F d \\
             & = \frac{1}{2} \left( \sigma \Delta x \Delta y \right) \left( \epsilon \Delta z \right) \\
             & = \frac{1}{2} \sigma \epsilon \Delta V
\end{align*}
where \(\Delta V\) is the volume of the element. If we consider the strain energy 
\textit{per unit volume}, then
\begin{equation*}
    u = \frac{\Delta U}{\Delta V} = \frac{1}{2} \sigma \epsilon
\end{equation*}
where \(u\) is the \textbf{strain energy density}. \(u\) can also be determined 
by finding the area under the stress-strain diagram, and hence has the units \unit{J.m^{-3}}.
\subsubsection{Modulus of Resilience}
When the stress in a material reaches the
proportional limit, the strain energy density 
is referred to as the modulus of resilience.
\begin{equation*}
    u_r = \frac{1}{2} \sigma_{pl} \epsilon_{pl} = \frac{1}{2} \frac{\sigma_{pl}^2}{E}
\end{equation*}
The modulus of resilience is also the area under the proportional region of the stress-strain 
diagram.
\begin{figure}[H]
    \centering
    \includegraphics[height = 6cm, keepaspectratio = true]{figures/modulus_of_resilience.pdf}
    \caption{Modulus of resilience \(u_r\).}
    % \label{}
\end{figure}
\subsubsection{Modulus of Toughness}
Another important property of a material is its modulus of toughness, \(u_t\). This 
quantity represents the entire area under the stress-strain diagram.
\begin{figure}[H]
    \centering
    \includegraphics[height = 6cm, keepaspectratio = true]{figures/modulus_of_toughness.pdf}
    \caption{Modulus of toughness \(u_t\).}
    % \label{}
\end{figure}
\end{document}
