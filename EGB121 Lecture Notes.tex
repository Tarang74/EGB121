%!TEX program = xelatex
\documentclass{article}
\usepackage{LaTeX-Submodule/template}

% Additional packages & macros

% Header and footer
\newcommand{\unitName}{Engineering Mechanics}
\newcommand{\unitTime}{Semester 1, 2022}
\newcommand{\unitCoordinator}{Dr Tuquabo Tesfamichael}
\newcommand{\documentAuthors}{\textsc{Tarang Janawalkar}}

\fancyhead[L]{\unitName}
\fancyhead[R]{\leftmark}
\fancyfoot[C]{\thepage}

% Copyright
\usepackage[
    type={CC},
    modifier={by-nc-sa},
    version={4.0},
    imagewidth={5em},
    hyphenation={raggedright}
]{doclicense}

\date{}

\begin{document}
%
\begin{titlepage}
    \vspace*{\fill}
    \begin{center}
        \LARGE{\textbf{\unitName}} \\[0.1in]
        \normalsize{\unitTime} \\[0.2in]
        \normalsize\textit{\unitCoordinator} \\[0.2in]
        \documentAuthors
    \end{center}
    \vspace*{\fill}
    \doclicenseThis
    \thispagestyle{empty}
\end{titlepage}
\newpage
%
\tableofcontents
\newpage
%
\section{Stress and Strain}
\subsection{External Forces}
Rigid bodies are subjected to external force and couple moment systems that result from the effects of gravitational,
electrical, magnetic, or contact forces. Contact forces can be surface, linear, or concentrated forces.
\subsubsection{Types of Forces}
\begin{itemize}
    \item Compressive (pushing)
    \item Tensile (pulling)
    \item Shear (sliding)
    \item Torsional (twisting)
    \item Biaxial tension
    \item Hydrostatic compression
    \item Bending (induces tension, compression and shear)
\end{itemize}
\subsection{Internal Loadings}
External forces cause internal loadings that occur in equal and opposite collinear pairs as stresses and strains.
Internal loading is associated with \textbf{stress} while \textbf{strain} is a measure of a body's deformation.

These loadings have no external effects on the body, and are not included on a \textbf{Free Body Diagram} (FBD) if
the entire body is considered.

To determine the forces in each member, we can use the method of sections to represent the internal loading as external forces.
\subsection{Internal Resultant Loadings}
Although the exact distribution of the internal loading may be \textit{unknown}, we can determine the
resultant force \(\symbf{F}_R\) and resultant moment \(\left( \symbf{M}_R \right)_O\) about a point \(O\) by applying the
equations of equilibrium
\begin{align*}
    \sum \symbf{F}   & = \symbf{0}  \\
    \sum \symbf{M}_O & = \symbf{0}.
\end{align*}
\subsubsection{In 3D}
In 3D, we can represent resultant loadings using four vectors acting over the sectioned area.
\begin{description}
    \item[Normal force] \(\symbf{N}\) force acting perpendicular to the area
    \item[Shear force] \(\symbf{V}\) force acting on an axis tangent to the area
    \item[Torsional moment] \(\symbf{T}\) rotation about the perpendicular axis
    \item[Bending moment] \(\symbf{M}\) rotation about an axis tangent to the area
\end{description}
\subsubsection{In 2D}
In 2D, the body is subjected to a coplanar system of forces, where \(\symbf{T} = \symbf{0}\).
\subsubsection{In 1D}
In 1D, the body is only subjected to axial forces, where \(\symbf{V} = \symbf{T} = \symbf{M} = \symbf{0}\).
\subsection{Stress}
The force and moment acting at a specific point on a sectioned area of a body represent the resultant effects of
the distribution of internal loading that acts over the sectioned area.
\begin{definition}[Stress]
    Consider the quotient of the force \(\Delta \symbf{F}\) over an area \(\Delta A\), then as the \(\Delta A \to 0\),
    so does \(\Delta \symbf{F}\), while the quotient approaches a finite limit. This quotient is called the stress at
    that point.
    \begin{equation*}
        \symbfit{\sigma} = \lim_{\Delta A \to 0} \frac{\Delta \symbf{F}}{\Delta A}
    \end{equation*}
    Here the normal and shear stresses can be expressed using \(\sigma_z\) and \(\tau_{zx}\) and \(\tau_{zy}\).
    \begin{align*}
        \sigma_z  & = \lim_{\Delta A \to 0} \frac{\Delta F_z}{\Delta A} \\
        \tau_{zx} & = \lim_{\Delta A \to 0} \frac{\Delta F_x}{\Delta A} \\
        \tau_{zy} & = \lim_{\Delta A \to 0} \frac{\Delta F_y}{\Delta A}
    \end{align*}

    Stress describes the intensity of the internal force acting on a specific region passing through a point.

    The unit for stress is Pascal where \qty{1}{Pa} or \qty{1}{N.m^{-2}} and \qty{1}{MPa} or \qty{1}{N.mm^{-2}}.
\end{definition}
\subsection{Average Normal Stress}
To determine the average stress distribution acting over a cross-sectional area of an axially loaded bar,
we assume that the material is both \textit{homogeneous} and \textit{isotropic}.
This means the load \(P\) applied through the centroid of the cross-sectional area
will cause the bar to deform uniformly throughout the central region of its length.

By passing a section through a bar, equilibrium requires the resultant normal force \(N\) at the section to be
equal to the external force \(P\). And because the material undergoes a uniform deformation, it is necessary
that the cross section be subjected to a constant normal stress distribution.

As a result, each small area \(\Delta A\) on the cross section is subjected
to a force \(\Delta N = \sigma \Delta A\), where the sum of these forces over the entire cross-sectional area
is \(P\). By letting \(\Delta A \to \odif{A} \) and therefore also \( \Delta N \to \odif{N} \), then as \(\sigma\)
is a constant, we have
\begin{align*}
    \int \odif{N} & = \int_A \sigma \odif{A} \\
    N             & = \sigma A
\end{align*}
Therefore
\begin{equation*}
    \sigma_{\mathrm{avg}} = \frac{N}{A}
\end{equation*}
where in this case \(N = P\).
\begin{theorem}[Equilibrium]
    For an uniaxially loaded body, the equation of force equilibrium gives
    \begin{align*}
        \sigma \left( \Delta A \right) - \sigma' \left( \Delta A \right) & = 0       \\
        \sigma                                                           & = \sigma'
    \end{align*}
    hence the normal stress components are must be equal in magnitude but opposite in direction.

    Under this condition, the material is subjected to \textbf{uniaxial stress} and this analysis
    applies to members subjected to tension or compression.
\end{theorem}
\subsection{Strain}
\begin{definition}[Deformation]
    Whenever a force is applied to a body, it will tend to change the body's shape and size.
    These changes are referred to as deformation.
\end{definition}
\begin{definition}[Strain]
    To describe the deformation of a body through changes in lengths of line segments on the surface, 
    we will develop the concept of strain.
    If an axial load \(P\) is applied to a bar, it will change the bar's length \(L_0\) to \(L\). 
    Then the \textbf{average normal strain} of the bar is defined
    \begin{equation*}
        \epsilon_{\mathrm{avg}} = \frac{L - L_0}{L_0}
    \end{equation*}
    where the numerator is often written as \(\delta = L - L_0\) and is known as elongation or extension.

    The \textbf{normal strain} \(\epsilon\) at a point in a body with an arbitrary shape is defined similarly.
    Consider a small line segment \(\Delta s\) which becomes \(\Delta s'\) after deformation. Then the limit of the normal strain
    is
    \begin{equation*}
        \epsilon = \lim_{\Delta s \to 0} \frac{\Delta s' - \Delta s}{\Delta s}
    \end{equation*}
    In both cases normal strain is positive when the initial length elongates, and negative when the length contracts.

    Strain is a dimensionless quantity sometimes expressed \unit{mm/mm^{2}} or \unit{m/m^{2}}, or as a percentage.
\end{definition}
\subsection{Tension and Compression Tests}
\end{document}
